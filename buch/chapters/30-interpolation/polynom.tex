%
% polynom.tex
%
% (c) 2020 Prof Dr Andreas Müller, Hochschule Rapeprswil
%
\section{Interpolationspolynom
\label{buch:section:interpolationspolynom}}
\rhead{Lagrange-Interpolationspolynome}
In diesem Abschnitt wird die folgende Aufgabe gelöst.
\begin{aufgabe}[Interpolationspolynom]
\index{Interpolationspolynom}%
Gegeben Stützstellen
\[
a=x_0<x_1<x_2<\dots < x_{n-1}<x_n=b
\]
und Funktionswerte $f_i, 0\le i\le n$, finde ein Polynom $l(x)$
mit der Eigenschaft $l(x_i)=f_i$ für alle $i=0,1,\dots n$.
\end{aufgabe}
Gegeben sind also $n+1$ Bedingungen, die das Polynom erfüllen muss.
Abgesehen von trivialen Fällen wie dem Null-Polynom, muss ein Polynom
im Allgemeinen mindestens den Grad $n$ haben, damit alle 
Bedingungen durch geeignete Wahl der $n+1$ Koeffizienten erfüllt werden können.
\index{Nullpolynom}%
Man könnte das Polynom nämlich in der Form
\[
p(x)
=
a_nx^n + a_{n-1}x^{n-1}+\dots+a_1x+a_0
\]
ansetzen und die Stützstellen einsetzen.
\index{Stützstellen}%
Lösung des Gleichungssystem
\index{Gleichungssystem!für das Interpolationspolynom}%
\begin{equation}
\begin{linsys}{5}
{\color{red}a_N}x_0^N &+& {\color{red}a_{N-1}}x_0^{N-1} &+& \dots &+& {\color{red}a_1}x_0 &+& {\color{red}a_0}x_0 &=& f_0 \\[5pt]
{\color{red}a_N}x_1^N &+& {\color{red}a_{N-1}}x_1^{N-1} &+& \dots &+& {\color{red}a_1}x_1 &+& {\color{red}a_0}x_1 &=& f_1 \\[5pt]
\vdots   & &    \vdots        & & \ddots& & \vdots & & \vdots & & \vdots\\[5pt]
{\color{red}a_N}x_n^N &+& {\color{red}a_{N-1}}x_n^{N-1} &+& \dots &+& {\color{red}a_1}x_n &+& {\color{red}a_0}x_n &=& f_n 
\end{linsys}
\end{equation}
für die Koeffizienten $\color{red}a_k$
liefert dann die gesuchten Koeffizienten.
\index{Koeffizient}%
Dieser Weg ist allerdings sehr aufwendig, die Lösung eines linearen
Gleichungssystems mit dem Gauss-Algorithmus benötigt $O(n^3)$ Operationen.
\index{Gauss-Algorithmus}%
Die sehr spezielle Struktur des Gleichungssystems sollte ermöglichen,
das Polynom $l(x)$ auf direkterem Weg zu ermitteln.

%
% Interpolationspolynom bestimmen
%
\subsection{Bestimmung des Interpolationspolynoms
\label{buch:section:interpolation:bestimmung}}
Das allgemeine Interpolationsproblem kann leicht gelöst werden, wenn
das folgende spezielle Interpolationsproblem gelöst ist.
\index{Interpolationspolynom!spezielles}%
\index{spezielles Interpolationspolynom}%

\begin{aufgabe}[Spezielle Interpolationspolynome]
\label{buch:aufgabe:speziellesinterpolationsproblem}
Gegeben die Stützstellen
\[
a=x_0<x_1<x_2<\dots <x_{n-1}<x_n=b,
\]
finde Polynome $l_j$ vom Grad $n$ derart, dass
\[
l_j(x_i) = \delta_{ij}=\begin{cases}
1&\qquad i=j\\
0&\qquad\text{sonst.}
\end{cases}
\]
\end{aufgabe}

\begin{figure}
\centering
\includegraphics{chapters/30-interpolation/figures/basis.pdf}
\caption{Polynome $l_j(x)$, welche des spezielle Interpolationsproblem
\ref{buch:aufgabe:speziellesinterpolationsproblem}
lösen.
\label{buch:figure:spezielleinterpolation}}
\end{figure}

Jedes der Interpolationspolynome $l_j$ hat Grad $n$, also hat auch eine
beliebige Linearkombination den Grad höchstens $n$.
\index{Linearkombination}%
Die Linearkombination
\[
p(x) = \sum_{j=0}^n f_j l_j(x)
\]
ist das gesuchte Interpolationspolynom, wie Einsetzen von $x_i$ in
\[
p(x_i)
=
\sum_{j=0}^n f_jl_j(x_i)
=
\sum_{j=0}^n f_j\delta_{ij}
=
f_i
\]
bestätigt.

\begin{beispiel}
Ein besonders einfacher Fall ist $n=1$.
Gesucht ist eine lineare Funktion $l(x)=a_1x+a_0$ derart, dass
$l(x_0)=f_0$ und $l(x_1)=f_1$.
Polynome $l_0$ und $l_1$ können leicht angegeben werden:
\[
l_0(x) = \frac{x_1-x}{x_1-x_0}
\qquad\text{und}\qquad
l_1(x) = \frac{x-x_0}{x_1-x_0}
\]
haben die geforderten Eigenschaften.
Die gesuchte Interpolationsfunktion ist daher
\[
p(x)
=
\frac{x_1-x}{x_1-x_0}f_0 + \frac{x-x_0}{x_1-x_0} f_1
=
x \frac{f_1-f_0}{x_1-x_0}   + \frac{x_1f_0-x_0f_1}{x_1-x_0}.
\]
Der Koeffizient von $x$ ist wie erwartet die Steigung der Geraden durch
die Punkte $(x_0,f_0)$ und $(x_1,f_1)$.
\end{beispiel}

Ein Polynom vom Grad $n+1$, welches in {\em allen} Stützstellen verschwindet,
ist leicht zu finden, es ist 
\[
l(x)
=
(x-x_0)(x-x_1)(x-x_2)\cdots (x-x_{n-1})(x-x_n).
\]
\index{lx@$l(x)$}%
Ein Polynom, welches nur an der Stützstelle $x_j$ {\em nicht} verschwindet,
ensteht, indem man den Faktor $(x-x_j)$ weglässt, es hat den Grad $n$.
Wir führen dafür die Notation
\[
(x-x_0)(x-x_1)(x-x_2)\cdots \widehat{(x-x_j)}\cdots (x-x_{n-1})(x-x_n),
\]
der Hut bedeutet, dass dieser Faktor weggelassen werden soll.
Allerdings hat dieses Polynom nicht den geforderten Wert $1$, man muss es
also noch mit einer geeigneten Konstante multiplizieren.
Das gesuchte Polynom $l_j(x)$ hat daher die Form
\[
l_j(x)
=
c_j(x-x_0)(x-x_1)(x-x_2)\cdots \widehat{(x-x_j)}\cdots (x-x_{n-1}(x-x_n).
\]
Einsetzen von $x_j$ ergibt
\[
l_j(x_j) = 1 = 
c_j(x_j-x_0)(x_j-x_1)(x_j-x_2)\cdots \widehat{(x_j-x_j)}\cdots(x_j-x_{n-1})(x_j-x_n),
\]
\index{ljx@$l_j(x)$}%
die Konstante $c_j$ ist daher
\[
c_j = \prod_{i=0\atop i\ne j}^n \frac{1}{x_j-x_i}.
\]

\begin{beispiel}
Man finde ein Polynome, welches $l(0)=l(1)=0$ und $l(\frac12)=1$
erfüllt.
Wegen $f_0=f_2=0$ ist nur das Polynom $l_1$ zu ermitteln, es ist
\[
l(x) = l_1(x)
=
\frac{(x-x_0)(x-x_2)}{(x_1-x_0)(x_1-x_2)}
=
\frac{x(x-1)}{\frac12(\frac12-1)}
=
\frac14x(1-x).
\qedhere
\]
\end{beispiel}

%
% Fehler des Interpolationspolynoms
%
\subsection{Fehler von Approximationspolynomen
\label{buch:section:interpolation:fehler}}
Getreu der Maxime, dass wir zu jeder numerischen Lösungsformel auch
Informationen über die zu erwartenden Fehler brauchen, entwickeln
wir in diesem Abschnitt die Theorie des Fehlers der Approximationspolynome.
\index{Fehler des Approximationspolynoms}%
Wir müssen zu diesem Zweck einen kleinen Ausflug in die Analysis unternehmen
in einen Bereich, der im Unterricht manchmal etwas zu kurz kommt.

Wenn die Ableitung einer Funktion in einem Intervall klein ist,
dann werden auch die Funktionswerte im Inneren dieses Intervalls
nicht gross von den Werten am Rand abweichen können.
Eine grosse Abweichung würde ja automatisch eine Steigung einer Sekanten
und damit auch eine grosse Steigung einer Tangenten zur Folge haben.
\index{Sekante}%
\index{Tangente}%
Dies ist die Idee, die den nachfolgend entwickelten Fehlerabschätzungen
zu Grunde liegt.
\index{Fehlerabschätzung}%

\subsubsection{Der Zwischenwertsatz}
\index{Zwischenwertsatz}%
Der Ausgangspunkt aller nachfolgenden Überlegungen ist die intuitiv
anschauliche Tatsache, dass eine stetige Funktion keine Sprünge macht.

\begin{satz}[Zwischenwertsatz]
Eine auf dem Intervall $[a,b]$ stetige Funktion nimmt jeden Wert im
Intervall $[f(a),f(b)]$ an.
Anders ausgedrückt, für jedes $y$ zwischen $f(a)$ und $f(b)$ gibt es ein 
$x$ zwischen $a$ und $b$ derart, dass $y=f(x)$.
\end{satz}

Dieser Satz war natürlich bereits die Grundlage des Verfahrens der
Intervallhalbierung, mit welchem wir in
Abschnitt~\ref{buch:subsection:intervallhalbierung}
Gleichungen gelöst haben.
\index{Intervallhalbierung}%
Wenn die Funktion an den Intervallenden verschiedene Vorzeichen hat,
dann muss es eine Nullstelle im Inneren des Intervalls geben.
\index{Nullstelle}%
Die Intervallhalbierung hat in jedem Schritt ein neues Intervall
konstruiert, das die Nullstelle enthielt.

\subsubsection{Der Satz von Rolle}
\begin{figure}
\centering
\includegraphics{chapters/30-interpolation/figures/rolle.pdf}
\caption{Satz von Rolle: eine nicht konstante differenzierbare Funktion,
die an den Enden eines Intervalls den gleichen Funktionswert hat, hat im 
Inneren des Intervalls eine Stelle $\xi$ mit Ableitung $0$.
\label{buch:figure:rolle}}
\end{figure}
Der Satz von Rolle erweitert den Zwischenwertsatz auf die Ableitung einer
differenzierbaren Funktion an (Abbildung~\ref{buch:figure:rolle}).

\begin{satz}[Rolle]
\label{buch:satz:rolle}
\index{Satz von Rolle}%
\index{Rolle, Satz von}%
Sei $f$ eine auf dem Intervall $[a,b]$ nicht konstante,
stetig differenzierbare Funktion
mit $f(a)=f(b)$, dann gibt es einen Punkt $\xi\in(a,b)]$ im Inneren
des derart, dass $f'(\xi)=0$.
\end{satz}

Der Satz von Rolle ist eine Selbstverständlichkeit, wenn die Ableitung
$f'(x)$ stetig ist, doch dies wird nicht vorausgesetzt, es wird nur
verlangt, dass die Ableitung existiert.
Ausserdem macht der Satz eine Aussage darüber, dass die Zwischenstelle
$\xi$ im Inneren des Intervalls sei.

\begin{proof}[Beweis]
Eine stetige Funktion hat auf dem kompakten Intervall $[a,b]$ mindestens
ein Maximum und ein Minimum.
Da die Funktion nicht konstant ist, ist das Maximum oder das Minimum
von $f(a)$ verschieden.
Wir nehmen an $\xi\in[a,b]$ sei ein Maximum mit dieser Eigenschaft,
das Argument für das Minimum ist völlig analog.
Wegen $f(\xi)>f(a)$ ist $\xi$ ein Punkt im Inneren des Intervalls,
also kann $\xi$ nicht $=a$ sein, folglich ist $\xi\in(a,b)$.

Wegen $f(\xi) \ge f(x)\forall x\in[a,b]$
folgt dann
\begin{align*}
f'(\xi) &= \lim_{h\to 0+}  \frac{f(\xi+h)-f(\xi)}{h} \le 0
\\
f'(\xi) &= \lim_{h\to 0-}  \frac{f(\xi+h)-f(\xi)}{h} \ge 0.
\end{align*}
Da $f$ differenzierbar ist, müssen diese beiden Grenzwerte übereinstimmen,
also ist $f'(\xi)=0$.
\end{proof}


\subsubsection{Nullstellen und der Satz von Rolle}
\begin{figure}
\centering
\includegraphics{chapters/30-interpolation/figures/nullstellen.pdf}
\caption{Schachtelung der Nullstellen von $f(x)$, $f'(x)$ und $f''(x)$.
Der Satz von Rolle~\ref{buch:satz:rolle} impliziert, dass sich zwischen zwei
Nullstellen von $f$ immer eine Nullstelle von $f'$ befindet, und
ebenso zwischen zwei Nullstellen von $f'$ eine von $f''$.
\label{buch:figure:nullstellen}}
\end{figure}
Ist $p(x)$ ein Interpolationspolynom für die Funktion $f(x)$, dann ist
$f(x_i)-p(x_i)=0$ für alle Stützstellen.
Insbesondere macht der Satz von Rolle Aussagen über 
die Differenz $f(x)-p(x)$ zwischen den Stützstellen.
Diese Frage wird im folgenden Satz genauer untersucht.

\begin{satz}
\label{buch:satz:nullstellen}
Ist $f$ eine differenzierbare Funktion auf dem Intervall $[a,b]$
mit Nullstellen 
\[
a=x_0 < x_1 < x_2 < \dots < x_{n-1} < x_n = b,
\]
die auf keinem Teilintervall $[x_i,x_{i+1}]$ konstant ist,
dann hat $f'$ im Inneren jedes Teilintervalls $[x_i, x_{i+1}]$
eine Nullstelle.
\end{satz}

Das Polynom 
\[
l(x) = (x-x_0)(x-x_1)\dots (x-x_{n-1})(x-x_n),
\]
welches für die Konstruktion des Interpolationspolynoms verwendet
wurde, hat genau die Nullstellen $x_0,x_1,\dots,x_{n-1},x_n$.
Nach dem Satz~\ref{buch:satz:nullstellen} muss es zwischen je
zwei aufeinanderfolgenden Nullstellen von $l$ eine Nullstelle der
Ableitung geben. 
Diese Situation ist in Abbildung~\ref{buch:figure:nullstellen}
für den Fall $l(x)=(x+2)(x+1)x(x-1)(x-2)$ dargestellt.

Die höheren Ableitungen $f^{(k)}$ haben ihre Nullstellen
natürlich auch wieder zwischen den Nullstellen der Ableitung $f^{(k-1)}$.
Die $n$-te Ableitung ist konstant und hat keine Nullstellen.

%\begin{figure}
%\centering
%\includegraphics{chapters/30-interpolation/figures/norm.pdf}
%\caption{Fehler des Lagrange-Interpolationspolynoms für die Funktion
%$f(x)=e^{-x^2/2}/\sqrt{2\pi}$.
%Der Fehler nimmt mit der Anzahl der Stützstellen bis $n=30$ ab, danach
%wird die Berechnung instabil und der Fehler nimmt wieder zu.
%\label{buch:figure:lagrangefehler}}
%\end{figure}

\subsubsection{Der Mittelwertsatz der Differentialrechnung}
\index{Mittelwertsatz}%
\index{Mittelwertsatz!der Differentialrechnung}%
Der Satz von Rolle sagt etwas über die Nullstellen von der Ableitung
einer differenzierbaren Funktion.
\index{Ableitung}%
Zwischen zwei Argumentwerten mit gleichem Funktionswert gibt es immer
eine Nullstelle der Ableitung.
Der Mittelwertsatz der Differentialrechnung verallgemeinert diese
Aussage auf verschiedene Funktionswerte.

\begin{figure}
\centering
\includegraphics{chapters/30-interpolation/figures/mittelwertsatz.pdf}
\caption{Der Mittelwertsatz~\ref{buch:satz:mittelwertsatz} besagt, dass
die Sekante des Graphen einer differenzierbaren Funktion immer eine
Tangente mit der gleichen Steigung hat.
Diese Eigenschaft kann man auch für eine beliebige ebene Kurven
erwarten (rechts), dies ist der Inhalt des verallgemeinerten
Mittelwertsatzes~\ref{buch:satz:vmittelwertsatz}.
\label{buch:polynome:figure:mittelwertsatz}}
\end{figure}

\begin{satz}[Mittelwertsatz]
\label{buch:satz:mittelwertsatz}
Zu einer auf dem Intervall $[a,b]$ differenzierbaren Funktion $f(x)$ gibt
es immer ein $\xi\in[a,b]$ mit
\[
f'(\xi) = \frac{f(b)-f(a)}{b-a}
\]
(Abbildung~\ref{buch:polynome:figure:mittelwertsatz} links).
\end{satz}

\begin{proof}[Beweis]
Man betrachtet die Funktion 
\[
g(x) = f(x) - \frac{f(b)-f(a)}{b-a}(x-a),
\]
sie hat an den Intervallenden die Werte
\begin{align*}
g(a) &= f(a) - \frac{f(b)-f(a)}{b-a}(a-a)=f(a),
\\
g(b) &= f(b) - \frac{f(b)-f(a)}{b-a}(b-a) = f(b) - \bigl(f(b)-f(a)\bigr) = f(a).
\end{align*}
Die Funktionswerte an den Intervallenden sind also gleich,
nach dem Satz~\ref{buch:satz:rolle} von Rolle hat also
die Ableitung $g'$ eine Nullstelle $\xi\in[a,b]$:
\[
g'(\xi) = f'(\xi) - \frac{f(b)-f(a)}{b-a} = 0
\quad
\Rightarrow
\qquad
f'(\xi) = \frac{f(b)-f(a)}{b-a}.
\qedhere
\]
\end{proof}

\subsubsection{Mittelwertsatz und ebene Kurven}
\index{ebene Kurve}%
\index{Kurve, ebene}%
Die von Abbildung~\ref{buch:polynome:figure:mittelwertsatz} vermittelte
Intuition lässt sich auch für beliebige parametrisierte Kurven
$\gamma\colon[a,b]\to\mathbb R:\to\mapsto (x(t),y(t))$ in der Ebene
verallgemeinern (Abbildung~\ref{buch:polynome:figure:mittelwertsatz} rechts).
Sind die Punkte $A=(x(a),y(a))$ und $B=(y(a),y(b))$ verschieden,
so dass die Richtung von $A$ nach $B$ wohldefiniert ist, dann erwarten
wir einen Zwischenstelle $\xi$ mit einer Tangente mit derselben Richtung.
\index{Tangente}%

\begin{satz}
\label{buch:satz:mittelwertsatz2d}
Sei $\gamma\colon[a,b]\to\mathbb R:t\mapsto (x(t),y(t))$ eine
stetig differenzierbare ebene Kurve mit $|\dot{\gamma}(t)|\ne 0$ für
alle $t\in[a,b]$ und $A=\gamma(a)\ne\gamma(b)=B$.
Dann gibt es ein $\xi\in[a,b]$ derart, dass $\dot{\gamma}(\xi)$ die gleiche
Richtung hat wie $\overrightarrow{AB}$.
\end{satz}

\begin{proof}[Beweis]
Zwei Vektoren haben die gleiche Richtung, wenn die Determinante
mit den Vektoren als Spaltenvektoren verschwindet.
\index{Determinante}%
Der Mittelpunkt der Strecke $AB$ ist
$M=(x_M,y_M) = (\frac12(x(a)+x(b)),\frac12(y(a)+y(b)))$.
Wir definieren die Funktion
\[
h(t)
=
\det(\overrightarrow{M\gamma(t)}, \overrightarrow{AB})
=
\biggl|
\begin{matrix}
x(t) - x_M & x(b)-x(a) \\
y(t) - y_M & y(b)-y(a)
\end{matrix}
\biggr|
\]
Da die Strecken $AB$ und $MA$ bzw.~$MB$ die gleiche Richtung haben,
gilt an den Stellen $t=a$ und $t=b$ 
\[
h(a)
=
\det (\overrightarrow{MA},\overrightarrow{AB})
= 0
\qquad\text{and}\qquad
h(b)
=
\det (\overrightarrow{MB},\overrightarrow{AB})
=
0.
\]
Nach dem Satz von Rolle muss es daher ein $\xi\in[a,b]$ geben mit
\index{Satz von Rolle}%
\begin{equation}
0
=
h'(\xi)
=
\det(
\dot{\gamma}(\xi),
\overrightarrow{AB}
)
=
\biggl|
\begin{matrix}
x'(\xi) & x(b)-x(a) \\
y'(\xi) & y(b)-y(a)
\end{matrix}
\biggr|,
\label{buch:polynome:eqn:mittelwertsatz2d}
\end{equation}
was natürlich wieder bedeutet, dass die Tangente $\dot{\gamma}(\xi)$ 
parallel ist zur Strecke $AB$.
\end{proof}


\begin{satz}[Verallgemeinerter Mittelwertsatz]
\label{buch:satz:vmittelwertsatz}
Sind $x(t)$ und $y(t)$ stetig differenzierbare Funktionen auf dem Intervall
$[a,b]$ derart, dass $x(a)\ne x(b)$ oder $y(a)\ne x(b)$ und ausserdem
$y'(t)\ne 0$ für $t\in[a,b]$.
Dann gibt es ein $\xi\in[a,b]$ mit
\[
\frac{y'(\xi)}{x'(\xi)}
=
\frac{y(b)-y(a)}{x(b)-x(a)}.
\]
\end{satz}

\begin{proof}[Beweis]
Die Bedingungen bedeuten, dass die die Kurve $\gamma(x(t),y(t))$
den Voraussetzungen des Satzes~\ref{buch:satz:mittelwertsatz2d}
genügt.
Es gibt daher einen Punkten $\xi\in[a,b]$ wo die Tangente
parallel zur Strecke $AB$ ist.
Nach~\ref{buch:polynome:eqn:mittelwertsatz2d} bedeutet dies
\[
y'(\xi) \cdot (x(b)-x(a))
-
x'(\xi) \cdot (y(b)-y(a))
=0
\qquad\Rightarrow\quad
\frac{x'(\xi)}{y'(\xi)}
=
\frac{y(b)-y(a)}{x(b)-x(a)}.
\qedhere
\]
\end{proof}

\subsubsection{Mittelwertsatz in $\mathbb R^n$}
\index{Mittelwertsatz in $\mathbb R^n$}
Der Mittelwertsatz verspricht die Existenz eines Arguments $\xi\in[a,b]$,
gibt aber keinerlei Hinweise darauf, wie $\xi$ berechnet werden könnte. 
Ausserdem lässt sich der Satz in dieser Form nicht auf höherdimensionale
Situationen verallgemeinern.
In vielen Anwendungen des Mittelwertsatzes ist der genaue Wert von $\xi$
gar nicht nötig.
Meistens wird nur verwendet, dass die Ableitung eine Abschätzung dafür
gibt, wie gross der Unterschied zwischen $f(b)$ und $f(a)$ sein kann.
\index{Ableitung}%
Dies kann zum Beispiel wie im folgenden Satz formuliert werden.

\begin{satz}
Für eine in $[a,b]$ stetig differenzierbare Funktion
$f\colon [a,b]\to\mathbb R^n$ gibt es ein $\xi\in[a,b]$ mit
\[
|f(b)-f(a)| \le |f'(\xi)|\cdot |b-a|.
\]
\end{satz}

\begin{proof}[Beweis]
\cite[(8.5.1)]{buch:dieudonne}
\end{proof}

Diese Form ist ausreichend, um zum Beispiel Fehlerabschätzungen in der
Numerik durchzuführen.
\index{Fehlerabschätzung}%

\subsubsection{Taylor-Reihe mit Lagrange-Restterm}
\index{Taylor-Reihe}%
\index{Lagrange-Restterm}%
Die Ableitung $f'(a)$ einer Funktion $f(x)$ an der Stelle $a$ ist definiert
als die beste lineare Approximation
\[
f(x) = f(a) + f'(a)\cdot (x-a) + o(|x-a|).
\]
Der Mittelwertsatz besagt, dass
\[
f(x) = f(a) + f'(\xi) \cdot (x-a)
\]
für ein geeignetes $\xi\in[a,x]$.
Man kann also sagen, dass $f(a)$ eine Approximation für $f(x)$ mit einem
Fehler der Form $f'(\xi)\cdot(x-a)$ für ein geeignetes $\xi\in[a,x]$ ist.
Das Taylor-Polynom verallgemeinert diese Beobachtung auf eine Approximation
höherer Ordnung.

\begin{satz}
Sei $f(x)$ eine $n+1$-mal stetig differenzierbare Funktion in einer
Umgebung von $a$.
Dann ist das {\em Taylor-Polynom}
\begin{align*}
T_{a,n}f(x)
&=
f(a) + f'(a)\cdot (x-a) + f''(x)\frac{(x-a)^2}{2!} + f'''(x)\frac{(x-a)^3}{3!}
+\dots+f^{(n)}(a)\frac{(x-a)^n}{n!}
\\
&=
\sum_{k=0}^n
f^{(k)}(a) \frac{(x-a)^k}{k!}
\end{align*}
eine Approximation von $f(x)$ derart, dass
\[
f(a)=T_{a,n}f(a),\quad
f'(a)=T_{a,n}f'(a),\quad
f''(a)=T_{a,n}f''(a),\quad
\dots,\quad
f^{(n)}(a)=T_{a,n}f^{(n)}(a).
\]
Ausserdem gibt es einen Wert $\xi\in[a,x]$ derart, dass der Fehler
\index{Fehler}%
\[
R_{n,a}f(x)
=
f(x) - T_{n,a}f(x)
=
f^{(n+1)}(\xi)\frac{(x-a)^{n+1}}{(n+1)!}
\]
ist.
\end{satz}
\index{Taylor-Polynom}%
$R_{n,a}f(x)$ heisst das {\em Lagrange-Restglied} des Taylor-Polynoms.
\index{Lagrange-Restglied}%

\begin{proof}[Beweis]
Zunächst kann die Behauptung über die Ableitungen von $T_{n,a}f(x)$ durch
die Rechnung
\index{Ableitung}%
\begin{align*}
T_{n,a}f'(x)
&=
f'(a) + f''(a) \cdot (x-a) + f'''(a)\frac{(x-a)^2}{2!} + \dots
&&\Rightarrow&
T_{n,a}f'(a) 
&=
f'(a)
\\
T_{n,a}f''(x)
&=
f''(a) + f'''(a) \cdot (x-a) + \dots
&&\Rightarrow&
T_{n,a}f''(a) 
&=
f''(a)
\\
&\;\vdots&&&&\;\vdots
\\
T_{n,a}f^{(n)}(x)
&=
f^{(n)}(a)
&&\Rightarrow&
T_{n,a}f^{(n)}(a)
&=
f^{(n)}(a)
\end{align*}
direkt bestätigt werden.
Daraus folgt, dass die Funktion $g(x) = f(x) - T_{n,a}f(x)$ die Eigenschaft
\begin{equation}
g(a) = g'(a) = g''(a) = \dots = g^{(n)}(a)=0
\label{buch:taylor:gabl}
\end{equation}
hat.
Es muss als nur gezeigt werden, dass es unter den Voraussetzungen
\eqref{buch:taylor:gabl} eine Zahl $\xi$ gibt derart, dass
\[
g(x) = g^{(n+1)}(\xi)\frac{(x-a)^{n+1}}{(n+1)!}
\]
gilt.
Der Fall $n=0$ ist der Mittelwertsatz~\ref{buch:satz:mittelwertsatz}.
\index{Mittelwertsatz}.

Für $n>0$ kann die Behauptung mit vollständiger Induktion bewiesen werden.
\index{Induktion, vollständige}%
\index{vollständige Induktion}%
Sei also bereits bewiesen, dass für eine Funktion, deren Ableitungen bis
zur Ordnung $n-1$ an der Stelle $a$ verschwinden, eine Zahl $\xi\in[a,x]$
existiert mit
\[
g(x) = g^{(n)}(\xi)\frac{(x-a)^n}{n!}
\]
und müssen jetzt zeigen, dass dies auch für $n+1$ gilt.

Wir müssen $g(x)$ mit $(x-a)^{n+1}$ vergleichen und untersuchen
dazu den Quotienten
\[
\frac{g(x)}{(x-a)^{n+1}}.
\]
Zur Abkürzung schreiben wir $h(x) = (x-a)^{n+1}$, es ist also der
Quotient 
\[
\frac{g(x)}{h(x)}
=
\frac{g(x)-g(a)}{h(x)-h(a)}
\]
zu berechnen.
Nach dem verallgemeinerten Mittelwertsatz~\ref{buch:satz:vmittelwertsatz}
gibt es $\xi_1\in[a,x]$ mit
\index{Mittelwertsatz}%
\[
\frac{g'(\xi_1)}{h'(\xi_1)}
=
\frac{g(x)-g(a)}{h(x)-h(a)}.
\]
Da die Ableitung $h'(x)=(n+1)(x-a)^n$ ist, folgt
\begin{equation}
\frac{g'(\xi_1)}{h'(\xi_1)}
=
\frac{g'(\xi_1)}{(n+1) (\xi_1-a)^n}
=
\frac{g(x)-g(a)}{h(x)-h(a)}
=
\frac{g(x)}{(x-a)^{n+1}}.
\label{buch:taylor:eqn1}
\end{equation}
Die Ableitungen der Ordnung $\le n-1$ der Funktion $g'(x)$ verschwinden
im Punkt $a$, nach Induktionsvoraussetzung gibt es also ein $\xi$ derart,
dass
\index{Induktionsvoraussetzung}%
\[
g'(\xi_1)
=
(g')^{(n)}(\xi) \frac{(\xi_1-a)^{n}}{n!}
=
g^{(n+1)}(\xi)\frac{(\xi_1-a)^n}{n!}
\]
gilt.
Setzt man dies in \eqref{buch:taylor:eqn1} ein, erhält man
\[
\frac{g(x)}{(x-a)^{n+1}}
=
\frac{g'(\xi_1)}{(n+1)(\xi_1-a)^n}
=
g^{(n+1)}(\xi)\frac{(\xi_1-a)^n}{n!} \frac{1}{(n+1)(\xi_1-a)^n}
=
g^{(n+1)}(\xi)\frac{1}{(n+1)!}.
\]
Durch Multiplikation mit $(x-a)^{n+1}$ erhalten wir
\[
g(x) = g^{(n+1)}(\xi)\frac{(x-a)^{n+1}}{(n+1)!}.
\]
Damit ist der Induktionsschritt vollzogen und die Restformel bewiesen.
\index{Induktionsschritt}%
\end{proof}

\begin{beispiel}
Die Funktion $f(x)=x^N$ hat an der Stelle $a=0$ verschwindende
Ableitungen bis zur Ordnung $N-1$.
Das Taylor-Polynom $T_{0,n}f(x)$ der Ordnung $n$ dieser Funktion
verschwindet daher für $n<N$.
\index{Taylor-Polynom}%
Nach der Restformel für das Taylor-Polynom gibt es Zahlen $\xi_n$ derart
\begin{align*}
R_{n,0}f(x)
=
f(x)
&=
x^N
= 
f^{(n+1)}(\xi_n) \frac{x^{n+1}}{(n+1)!}
\\
&=
\frac{
N\cdot(N-1)\cdot(N-2)\cdot\dots\cdot(N-n)
}{
1\cdot 2 \cdot 3 \cdot\dots \cdot (n+1)
}
\xi_n^{N-n-1} 
=
\binom{N}{n+1} \xi_n^{N-n-1}
\\
\Rightarrow\qquad
\xi_n
&=
\binom{N}{n+1}^{\frac{-1}{N-n-1}} x^{\frac{N}{N-n-1}}.
\qedhere
\end{align*}
\end{beispiel}

\subsubsection{Fehlerabschätzungen}
\index{Fehlerabschätzung}%
Im Hinblick auf die nachfolgende Diskussion des Fehlers des
Interpolationspolynoms formulieren wir dies noch als eine
Fehlerabschätzung.
Das Taylor-Polynom approximiert die Funktion $f(x)$ durch ein
Polynom vom Grad $n$.
Der Fehler des Taylor-Polynoms ist
\[
|f(x) - T_{n,a}f(x)|
=
\biggl|
f^{(n+1)}(\xi)
\frac{(x-a)^{n+1}}{(n+1)!}
\biggr|
=
|f^{(n+1)}(\xi)|\cdot
\biggl|
\frac{(x-a)^{n+1}}{(n+1)!}
\biggr|
\le
\max_{\xi\in [a,x]} |f^{(n+1)}(\xi)|
\biggl|
\frac{(x-a)^{n+1}}{(n+1)!}
\biggr|,
\]
er ist also beschränkt einerseits durch ein Polynom mit Grad $n+1$
und den grössten Wert der $(n+1)$-te Ableitung im Intervall $[a,x]$.
Die Fehlerformel für das Interpolationspolynom wird von der genau
gleichen Art sein.
\index{Interpolationspolynom}%

\subsubsection{Fehler des Lagrange-Interpolationspolynoms}
\index{Lagrange-Interpolationspolynom}%
Der folgende Satz gibt vollständige Auskunft über den Fehler des
Interpolationspolynoms.

\begin{satz}
\label{buch:satz:lagrangefehler}
Sei $p$ ein Polynom vom Grad $n$, welches mit der $n+1$-mal differenzierbaren
Funktion $f$ an den $n+1$ Stellen
\[
a = x_0 < x_1 < x_2 < \dots  < x_{n-1} < x_n=b
\]
übereinstimmt.
Dann gibt es für jedes $x\in[a,b]$ ein $\xi_x\in [a,b]$ mit
\begin{equation}
f(x) - p(x) = \frac{f^{(n+1)}(\xi_x)}{(n+1)!} l(x).
\label{buch:equation:polyfehler}
\end{equation}
\end{satz}

\begin{proof}[Beweis]
An den Stützstellen $x_i$ ist $f(x_i)-p(x_i)=0$ und $l(x_i)=0$, die
Gleichung~\eqref{buch:equation:polyfehler} ist also trivialerweise
erfüllt.

Sei jetzt also $x\in[a,b]$ verschieden von allen $x_i$.
Da $l(x)\ne 0$ ist, gibt es eine Zahl $c$ derart, dass
\begin{equation}
f(x)-p(x)=cl(x)
\qquad\Leftrightarrow\qquad
f(x)-p(x)-cl(x)=0.
\label{buch:equation:polyfehler1}
\end{equation}
Die Funktion $g(x)=f(x)-p(x)-cl(x)$ verschwindet in allen Stützstellen $x_i$
und zusätzlich auch noch im Punkt $x$, sie hat also $n+1$ Nullstellen.

Nach dem Nullstellen-Schachtelungssatz~\ref{buch:satz:nullstellen}
hat die $n+1$-te Ableitung von $g$ eine Nullstelle im Intervall.
Es gibt also eine Zahl $\xi_x\in[a,b]$ mit $g^{(n+1)}(\xi_x)=0$.

Da $p$ Grad $n$ hat, ist die $n+1$-te Ableitung $0$.
Das Polynom $l(x)$ hat die Form
\[
l(x) = x^{n+1} -(x_0+x_1+\dots+x_{n-1}+x_n)x^{n-1} + \dots + (-1)^{n+1}x_0x_1\dots x_{n-1}x_n,
\]
seine $n+1$-Ableitung ist die Konstanten $(n+1)!$.

Die Folgerung $g^{(n+1)}(\xi_x)=0$ wird damit zu
\[
0 = f^{(n+1)}(\xi_x) -c (n+1)!
\qquad\Rightarrow\quad
c=\frac{f^{(n+1)}(\xi_x)}{(n+1)!}.
\]
Einsetzen diese Wertes für $c$ in \eqref{buch:equation:polyfehler1} ergibt
\[
f(x)-p(x) = cl(x)=\frac{f^{(n+1)}(\xi_x)}{(n+1)!} l(x),
\]
wie behauptet.
\end{proof}

Dieser Satz erlaubt den Fehler eines Interpolationspolynoms abzuschätzen,
wenn die $n+1$-te Ableitung der Funktion $f$ bekannt ist.
Wir bezeichnen mit
\[
\|g\| = \sup_{a\le x\le b} |g(x)|
\]
die {\em Supremun-Norm} der Funktion $g$ im Intervall $[a,b]$.
\index{Supremum-Norm}

\begin{korollar}
\label{buch:korollar:interpolationsfehler}
Ist $p$ ein Interpolationspolynom vom Grad $n$, welches mit der Funktion
$f$ in den Stellen $a=x_0<x_1<\dots <x_{n+1}<x_n=b$ übereinstimmt, dann
ist 
\[
|f(x)-p(x)| \le \frac{\|f^{(n+1)}\|}{(n+1)!} |l(x)|.
\]
\end{korollar}

Der Fehler des Interpolationspolynoms vom Grad $n$ ist also beschränkt
durch das Polynom $l(x)$ vom Grad $n+1$ und den grössten Wert der
$(n+1)$-ten Ableitung von $f(x)$.

\begin{beispiel}
\begin{figure}
\centering
\includegraphics{chapters/30-interpolation/figures/sin.pdf}
\caption{Interpolation der Funktion $f(x)=\sin x$ mit nur drei 
Stützstellen $x_0=0$, $x_1=\frac{\pi}2$ und $x_2=\pi$.
Der Fehler ist deutlich kleiner als die Abschätzung mit
Satz~\ref{buch:satz:lagrangefehler} erwarten lässt.
\label{buch:figure:sin}}
\end{figure}
Die Funktion $f(x)=\sin x$ soll mit den Stützstellen $x_0=0$, $x_1=\frac{\pi}2$
und $x_2=\pi$ interpoliert werden.
Das Interpolationspolynom ist ein quadratisches Polynom mit Nullstellen
$x_0$ und $x_2$, der Funktionswert bei $x_1$ muss $1$ sein.
Man kann sich davon überzeugen, dass das Polynom
\[
p(x) = \frac{4}{\pi^2} x(\pi -x )
\]
diese Eigenschaft hat (Abbildung~\ref{buch:figure:sin}).
Wie gross ist der Fehler dieses Interpolationspolynoms?

Die dritten Ableitungen der Funktion $f(x)=\sin x$ sind, bekannt, es ist
$f^{(3)}(x)=-\cos x$.
Der Betrag von $f^{(3)}(x)$ wird also nie grösser als $1$.
Es folgt, dass
\[
|f(x)-p(x)| \le \frac{1}{3!} l(x)
=
\frac16 |x(x-{\textstyle\frac{\pi}2})(x-\pi)|
\]
Die Ableitung des Polynoms auf der rechten Seite hat Nullstellen bei
$\frac{\pi}2 \pm \frac{\pi}{2\sqrt{3}}$,
durch Einsetzen erhält man den maximalen Wert
\[
\|f^{(3)}\|
=
\frac{\pi^3}{12\sqrt{3}}\approx 1.49179.
\]
Wir schliessen, dass das Interpolationspolynom niemals um mehr als $0.24863$
vom Funktionswert abweichen kann.
\end{beispiel}

%
% Runges Phänomen
%
\subsection{Runges Phänomen
\label{buch:section:interpolation:runge}}
\index{Runge}%
\index{Runges Phänomen}%
Das Korollar~\ref{buch:korollar:interpolationsfehler} besagt, dass der
Fehler des Interpolationspolynom unter anderem durch den Betrag von $l(x)$
begrenzt ist.
Für äquidistante Stützstellen mit Abstand $h$ kann man beobachten,
dass die Oszillationen des Polynoms $l(x)$ gegen den Rand des Intervalls
immer grösser werden.
\index{Oszillation}%
Für einen Punkte in der Mitte jedes Teilintervalls ist $h/2$ der kleinste
mögliche Faktor in $l(x)$. 
Den grössten Faktor findet man für $x$ im ersten oder letzten Teilintervall,
er ist $b-a-h$.
Ausserdem treten mehrere ähnlich grosse Faktoren auf.
Für $x$ in einem Intervall nahe $(a+b)/2$ sind die Faktoren
dagegen nur halb so gross.
Diese Phänomen, dass das die Amplitude der Oszillationen des
Interpolationspolynoms $l(x)$ gegen den Rand des Intervalls immer
grösser wird, ist auch als {\em Runges Phänomen} bekannt.

%
% Tschebyscheff Interpolation
%
\subsection{Wahl der Stützstellen und Tschebyscheff-Interpolationspolynom
\label{buch:section:interpolation:tschebyscheff}}
Die als Runges Phänomen beschriebenen Oszillationen am Rande des Intervalls
können verkleinert werden, wenn man dafür sorgt, dass
mit grösserem Abstand von der Mitte des Intervalls der Abstand der
Stützstellen untereinander ebenfalls verkleinert.
\index{Runges Phänomen}%
\index{Oszillation}%
Dies garantiert, dass neben den grossen Faktoren in der Nähe von $b-a$ 
auch wesentlich kleinere Faktoren auftreten, so dass die extremen Werte
nahe den Intervallenden vergleichbar mit den Werten im Inneren des
Intervalls werden.

\begin{figure}
\centering
\includegraphics[width=\hsize]{chapters/30-interpolation/figures/lissajous.pdf}
\caption{Diese Lissajous-Figur suggeriert eine mögliche Lösung für eine 
Interpolationspolynom mit besonders kleinem Fehler.
\index{Lissajous-Figur}%
Wenn sich diese Kurve als Polynom ausdrücken lässt, bleibt der Fehler über
das ganze Definitionsintervall gleichmässig beschränkt.
\label{buch:figure:lissajous}}
\end{figure}

\begin{figure}
\centering
\includegraphics[width=\hsize]{chapters/30-interpolation/figures/lissajous-chebyshef.pdf}
\caption{Lissajous-Figur von Abbildung~\ref{buch:figure:lissajous}
mit eingezeichnetem Tschebyscheff-Polynom und Nullstellen
desselben.
\index{Tschebyscheff-Polynom}%
\index{Nullstelle}%
Das Maximum des Betrags des Tschebyscheff-Polynoms ist 1 und liefert 
damit eine obere Schranke für den Fehler des zugehörigen
Interpolationspolynoms.
\index{Fehler}%
\label{buch:figure:lissajous-chebyshef}}
\end{figure}

Die beste Approximation durch ein Interpolationspolynom kann man also
erwarten, wenn $l(x)$ im Intervall $[a,b]$ keine besonders grossen Werte
annimmt.
\index{Approximation}%
Eine Funktion ähnlich wie $\sin x$ oder $\cos x$, die unendlich viele
Extremewerte $\pm 1$ haben, würde dieses Kriterium erfüllen, aber
$\sin x$ und $\cos x$ sind keine Polynome.
Sie sind auch auf auf einem viel grösseren Intervall als nötig definiert,
nämlich ganz $\mathbb R$.
Ein verwandtes Beispiel sind Lissajous-Figuren.
\index{Lissajous-Figur}%
Abbildung~\ref{buch:figure:lissajous} suggeriert, dass eine geeignete
Lissajous-Figur als Graph für ein Interpolationspolynom mit sehr
geringem Fehler dienen könnte, wenn man sie als Polynom darstellen kann.
Eine solche Lissajous-Figur entsteht als Kurve
$\gamma_n\colon t\mapsto (\cos t, \cos nt)$ oder eine anderes trigonometrisches
Polynom als zweite Komponente.
\index{trigonometrisches Polynom}%
\index{Polynom!trigonometrisch}%
Die Abbildung~\ref{buch:figure:lissajous-chebyshef} zeigt diese Kurve
$\gamma_n(t)$
für den Fall $n=13$ der Lissajous-Figur von
Abbildung~\ref{buch:figure:lissajous} überlagert.
Die gute Übereinstimmung bestätigt die obige Beobachtung.

\subsubsection{Die Tschebyscheff-Polynome}
Es stellt sich also die Frage, ob $\cos nt$ als Polynom in $z=\cos t$
ausgedrückt werden kann.
Sei also
\[
T_n(x)
=
T_n(\cos t)
= 
\cos nt.
\]
Für kleine $n$ kann man unmittelbar verifizieren, dass $T_n(z)$ ein
Polynom ist:
\begin{equation}
\begin{aligned}
T_0(x) &= 1,\\
T_1(x) &= x,\\
T_2(x) &= \cos 2t = 2\cos^2 t-1 = 2x^2 -1
\quad\text{und}
\\
T_3(x) &= \cos 3t = 4\cos^3 t - 3\cos t = 4x^3-3x.
\end{aligned}
\label{buch:tschebyscheff:erste}
\end{equation}
Für kleine Werte von $n$ ist also direkt nachprüfbar, dass $T_n(x)$
ein Polynom ist.

Wir benötigen eine Formel, mit der sich Werte der Polynome $T_n(x)$
effizient berechnen lassen, ohne dass das Polynom in expliziter
Form ermittelt werden muss.
Im Folgenden soll dazu eine Rekursionsformel hergeleitet werden.
\index{Rekursionsformel}%

Aus den Additionstheoremen für den Kosinus folgt die Formel für die
Summe von zwei Konsinus-Funktionen
\index{Additionstheorem}%
\begin{align}
\cos (n+1)t + \cos(n-1)t
&=
2 \cos \frac{(n+1)t + (n-1)t}2 \cos \frac{(n+1)t -(n-1)t}2
=
2\cos nt \cos t
\notag
\\
T_{n+1}(x)+T_{n-1}(x)&=2T_n(x)x
\notag
\\
T_{n+1}(x) &= 2xT_n(x) - T_{n-1}(x).
\label{buch:tschebyscheff:rekursion}
\end{align}
Wenn $T_n(x)$ und $T_{n-1}(x)$ Polynome sind, dann ist auch $T_{n+1}(x)$
ein Polynom.
Aus den bereits bekannten Polynomen~\eqref{buch:tschebyscheff:erste} und
der Rekursionsformel~\eqref{buch:tschebyscheff:rekursion} folgt jetzt mit
vollständiger Induktion, dass alle $T_n(x)$ Polynome sind.
\index{Induktion}%
\index{vollständige Induktion}%
Sie heissen {\em Tschebyscheff-Polynome}.
\index{Tschebyscheff-Polynom}%
Die Rekursionsformel kann dazu verwendet werden, die Polynome explizit
zu berechnen.
Zum Beispiel folgt für die nächsten drei Polynome
\begin{align*}
T_4(x) &= 8x^4-8x^2 + 1,
\\
T_5(x) &= 16x^5-20x^3+5x \quad\text{und}
\\
T_6(x) &= 32x^6-48x^4+18x^2-1.
\end{align*}
Da das Interpolationspolynom den führenden Koeffizienten $1$ haben muss,
muss $l(x) = 2^{1-n}T_n(x)$ gewählt werden.

\subsubsection{Tschebyscheff-Stützstellen}
\index{Tschebyscheff-Stützstellen}%
Die Polynome $T_n(x)$ sind eigentlich nicht nötig, da für die Konstruktion
des Interpolationspolynoms nur die Nullstellen nötig sind.
Wegen $T_n(x)=\cos nt$ liegt eine Nullstelle genau dann vor, wenn
$nt = \frac{\pi}2 + k\pi$, $k\in\mathbb Z$.
Die zugehörigen Werte von $z$ sind
\[
x_k
=
\cos t = \cos\frac{\pi(2k+1)}{2n}.
\]
In Abbildung~\ref{buch:figure:tschebyscheff-vergleich} sind 
die Polynome $2^{n-1}l(x)$ vom Grad $n$ oben für äquidistante Stützstellen
und unten für Tschebyscheff-Stützstellen im Vergleich dargestellt.
Wie in der Einleitung zu diesem Abschnitt angekündigt, oszillieren die
Polynome für äquidistante Stütztstellen nahe den Intervallenden.
\index{Oszillation}%
Für Tschebyscheff-Stützstellen wird $2^{n-1}l(x)$ betragsmässig nie
grösser als $1$.
\begin{figure}
\centering
\includegraphics{chapters/30-interpolation/figures/vergleich.pdf}
\caption{Vergleich der Oszillationen bei äquidistanten Stützstellen (oben)
und bei Tschebyscheff-Stützstellen.
Damit die Abweichungen sichtbar werden, sind die Polynome $l(x)$ vom Grad
$n$ mit dem Faktor $2^{n-1}$ skaliert.
Bei Verwendung von Tscheby\-scheff-Stützstellen wächst $2^{n-1}l(x)$
nie über $1$ an, während bei äquidistanten Stützstellen die in der
Einleitung zu diesem Abschnitt diskutierten Oszillationen nahe den
Intervallenden auftreten.
\label{buch:figure:tschebyscheff-vergleich}}
\end{figure}

\begin{figure}
\centering
\includegraphics{chapters/30-interpolation/figures/tschebasis.pdf}
\caption{Basisinterpolationspolynome vom Grad 7 für
Tschebyscheff-Stützstellen
\label{buch:figure:tschebyscheffbasis}}
\end{figure}
Abbildung~\ref{buch:figure:tschebyscheffbasis} zeigt die Basispolynome
vom Grad 7 $l_j(x)$ für Tschebscheff-Stütztstellen.
Da bei Verwendung von Tschebyscheff-Stützstellen die Polynome $l(x)$
keine ausgeprägten Oszillationen an den Intervallenden aufweisen, 
sind auch die Basispolynome $l_j(x)$ vor allem in der Nähe der
jeweiligen Stützstelle $x_j$ wesentlich von $0$ verschieden.


%
% normal.tex
%
% (c) 2020 Prof Dr Andreas Müller, Hochschule Rapperswil
%
\subsubsection{Vergleich der Fehler}
In diesem Abschnitt vergleichen wir die Fehler der Interpolationspolynome
für die Funktion
\index{Standardnormalverteilungsdichte}%
\index{Normalverteilung}%
\[
f(x)
=
\frac{1}{\sqrt{2\pi}} e^{-x^2/2}
\]
nach Lagrange mit äquidistanten Stützstellen und mit
\index{aquidistante stutzstellen@äquidistante Stützstellen}%
\index{Stützstellen!äquidistant}%
Tschebyscheff-Stützstellen.
\index{Tschebyscheff-Stützstellen}%

\begin{figure}
\centering
\includegraphics{chapters/30-interpolation/figures/norm.pdf}
\caption{Fehler des Lagrange-Interpolationspolynoms für die Funktion
$f(x)=e^{-x^2/2}/\sqrt{2\pi}$.
Der Fehler nimmt mit der Anzahl der Stützstellen bis $n=30$ ab, danach
wird die Berechnung instabil und der Fehler nimmt wieder zu.
\index{instabil}%
\label{buch:figure:lagrangefehler}}
\end{figure}


\begin{figure}
\centering
\includegraphics{chapters/30-interpolation/figures/tscheb.pdf}
\caption{Fehler des Interpolationspolynomes für die Funktion
$f(x)=e^{-x^2/2}/\sqrt{2\pi}$ mit Stützstellen nach Tschebyscheff.
Der Fehler bleibt über das ganze Intervall gleichmässig.
Für eine grosse Zahl von Stützstellen erreicht die Interpolation die
Maschinengenauigkeit.
\label{buch:figure:tschebyschefffehler}}
\end{figure}

Die Resultate sind in Abbildung~\ref{buch:figure:lagrangefehler}
für das Lagrange-Interpolationspolynom und
Abbildung~\ref{buch:figure:tschebyschefffehler}
für das Tschebyscheff-Interpolationspolynom dargestellt.
In beiden Abbildungen wird die gleiche Anzahl Stützstellen verwendet.
Beim Lagrange-Interpolationspolynom nimmt der Fehler zunächst schnell ab.
Er ist, wie das Runge Phänomen erwarten lässt, immer am grössten im
äussersten Teilintervall.
Für grösser werdende Anzahl von Stützstellen wird die Berechnung des 
Interpolationspolynoms in diesen äussersten Teilintervallen instabil
und wächst zum Beispiel von $n=30$ zu $n=34$ um eine Grössenordnung an.
Für eine grosse Zahl von Stützstellen kann das Lagrange-Interpolationspolynom
also mindestens am Rand des Intervalls keine zuverlässigen Approximation sein.

Die Verwendung von Tschebyscheff-Stützstellen verbessert die Genauigkeit
schon bei $n=16$ Stützstellen um zwei Grössenordnungen.
Aussserdem sind die Fehler über das ganze Intervall gleichmässig
verteilt.
Bei einer grösseren Anzahl von Stützstellen scheint der Fehler 
vor allem linken Rand stark anzuwachsen.
Da die Funktionswerte aber ungefähr $1$ sind, ist
ein Fehler von $10^{-16}$ genau der typische Rundungsfehler des
\texttt{double} Datentyps.
Man sieht hier als nicht den Fehler des Interpolationspolynoms sondern
die Grenzen der Maschinengenauigkeit.











