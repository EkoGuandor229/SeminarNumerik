%
% problemstellung.tex -- Beispiel-File für die Beschreibung des Problems
%
% (c) 2020 Prof Dr Andreas Müller, Hochschule Rapperswil
%
\section{Folgerungen
\label{ew:section:folgerungen}}
\rhead{Folgerungen}

Die Störungstheorie für Eigenwerte ist eine elegante Methode um kleine Änderungen eines Eigenwertproblems zu approximieren.
Da die Berechnung im nicht entarteten Fall mit wenigen Matrixmultiplikationen explizit durchgeführt werden kann, ist ein Bruchteil der Rechenleistung notwendig.

% \section{Anwendungsgebiet mit entarteten Eigenwerten}

% Aufspaltung der Energieniveaus von Atomen in der Quantenmechanik
% \begin{center}
%     \includegraphics[scale=0.25, clip, trim=0 85 0 0]{img/WasserstoffAufspaltung.pdf}
%     \footnote{Quantenmechanik, Mathematisches Seminar, Andreas Müller et. al (CC0 1.0)}
% \end{center}
