%
% problemstellung.tex -- Beispiel-File für die Beschreibung des Problems
%
% (c) 2020 Severin Weiss

\section{Folgerungen
\label{laplace:section:folgerungen}}
\rhead{Folgerungen}
Damit die Approximation von $\tilde{f}(t)$ möglichst genau $f(t)$
repräsentiert,
müssen die passenden Werte für $\lambda, \sigma $ und $\nu $ gefunden werden. 
Dies geschah in diesem Falle rein iterativ durch Probieren.
Insbesondere wurden folgende Funktionen ausgewertet $
F_{1}(s)=\frac{1}{1-s} $ und $F_{2}(s) = \frac{2}{s^{3}}$.
Die Werte, welche für die Auswertung verwendet wurden, sind in der Tabelle~\ref{laplace:parametertabelle2} abgebildet.

%\begin{table}
%\centering
%\begin{tabular}[c]{c|c|c}
%& $F_{1}(s)=\frac{1}{1-s}$ & $F_{2}(s) = \frac{2}{s^{3}}$ \\
%\hline
%Parameter & $\lambda=1.288$ & $\lambda=0.101$ \\
%& $\sigma=1.000001$ & $\sigma=0.965$ \\
%& $\nu=0.81$ & $\nu=0.098953$ \\
%\end{tabular}
%\caption{Funktionen im Zeitbereich und iterativ bestimmte Parameter $\lambda, \sigma$ und $\mu$ für die Methode von Talbot
%\label{laplace:parametertabelle1}}
%\end{table}
%Die Abbildungen \ref{laplace:fehlerf1} -- \ref{laplace:fehlerf2} zeigen die Verläufe der absoluten Fehler.

\begin{figure}
\centering
\includegraphics[width=6.9cm]{papers/laplace/Error_1divide_sminus1}
\includegraphics[width=6.9cm]{papers/laplace/Error_1divide_sminus1_bis_tgleich5}
\caption{Fehlerentwicklung von $F_{1}(s)$
\label{laplace:fehlerf1}
}
\end{figure}

%\begin{figure}
%\centering
%\caption{Fehler von $F_{1}(s)$
%\label{laplace:fehlerf1-5}
%}
%\end{figure}

\begin{figure}
\centering
\includegraphics[width=6.9cm]{papers/laplace/Error_2divide_s_pow3}
\includegraphics[width=6.9cm]{papers/laplace/Error_2divide_s_pow3_bis_tgleich5}
\caption{Fehlerentwicklung von $F_{2}(s)$
\label{laplace:fehlerf2}
}
\end{figure}

%\begin{figure}
%\centering
%\includegraphics[width=8cm]{papers/laplace/Error_2divide_s_pow3_bis_tgleich5}
%\caption{Fehler von $F_{2}(s)$
%\label{laplace:fehlerf2-5}
%}
%\end{figure}


Es ist deutlich ersichtlich, dass der Fehler nur für ein gewisses
Zeitintervall akzeptabel ist.
Im Bereich, wo der Fehler sich in einem
tolerierbaren Bereich befindet, wurden die Parameter $\lambda$,
$\sigma$ und $\nu$ für ein bestimmtes $t_{0}$ ermittelt.
Diese Parameter sind für spätere Zeitpunkte nicht mehr sinnvoll (der Fehler wird sehr gross).
Zu späteren Zeitpunkten müssten die Parameter auf ein Neues iterativ
ermittelt werden.
Der Versuch die optimierten Parameter durch einen Algorithmus zu
finden, ist in \cite{laplace:talbot} für Interssierte nachzuschlagen.
Für den Zeitpunkt $t_{0}=1$ wurden die Fehler in der Grössenordnung
$10^{-8}$ respektive $10^{-6}$ erreicht.
Dies geschah mittels Erhöhung der Iterationszahl $n$. Um noch bessere
Resultate zu erhalten, könnte die Romberg-Beschleunigung verwendet
\index{Romberg-Beschleunigung}%
werden (für detaillierte Information siehe Kapitel 4.2).

\begin{table}
\centering
\begin{tabular}{l|r@{\hskip2pt}l|r@{\hskip2pt}l}
& \multicolumn{2}{c|}{$F_{1}(s)$} & \multicolumn{2}{c}{$F_{2}(s)$}\\
\hline
Parameter & $\lambda$&$=1.288$ & $\lambda$&$=0.101$ \\
 & $\sigma$ &$=1.000001$ & $\sigma$&$=0.965$ \\
 & $\nu$ & $=0.81$ & $\nu$ & $=0.098953$ \\
\hline
$t_{0}$ & \multicolumn{2}{c|}{$1$} & \multicolumn{2}{c}{$1$} \\
\hline
Iterationszahl n & \multicolumn{2}{c|}{$63000$} & \multicolumn{2}{c}{$100000$} \\
\hline
Fehler & \multicolumn{2}{c|}{$1.7076~\cdot~10^{-08}$} & \multicolumn{2}{c}{$1.8447~\cdot~10^{-06}$} \\
\end{tabular}
\caption{Fehler zum Zeitpunkt $t_{0}$ mit iterativ optimierten Talbot-Parametern $\lambda, \sigma$ und $\mu$
\label{laplace:parametertabelle2}}
\end{table}


%\lstinputlisting[style=Python,linerange={1-50},firstnumber=1,float,caption={Implementierung Talbotmethde in Python},label={laplace:codes:talbotmethode}]{papers/laplace/TalbotsMethode_homemade_f_t.py}
%\lstinputlisting[style=Python,linerange={51-100},firstnumber=51,float,caption={Implementierung Talbotmethde in Python},label={laplace:codes:talbotmethode}]{papers/laplace/TalbotsMethode_homemade_f_t.py}
%\lstinputlisting[style=Python,linerange={101-150},firstnumber=101,float,caption={Implementierung Talbotmethde in Python},label={laplace:codes:talbotmethode}]{papers/laplace/TalbotsMethode_homemade_f_t.py}
%\lstinputlisting[style=Python,linerange={151-200},firstnumber=151,float,caption={Implementierung Talbotmethde in Python},label={laplace:codes:talbotmethode}]{papers/laplace/TalbotsMethode_homemade_f_t.py}
%\lstinputlisting[style=Python,linerange={201-250},firstnumber=201,float,caption={Implementierung Talbotmethde in Python},label={laplace:codes:talbotmethode}]{papers/laplace/TalbotsMethode_homemade_f_t.py}
%\lstinputlisting[style=Python,linerange={251-287},firstnumber=251,float,caption={Implementierung Talbotmethde in Python},label={laplace:codes:talbotmethode}]{papers/laplace/TalbotsMethode_homemade_f_t.py}

\FloatBarrier
\section{Programmcode\label{laplace:section:programmcode}}
\rhead{Programmcode}
\lstinputlisting[style=Python]{papers/laplace/TalbotsMethode_homemade_f_t.py}
