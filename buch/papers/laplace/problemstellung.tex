%
% problemstellung.tex -- Beispiel-File für die Beschreibung des Problems
%
% (c) 2020 Severin Weiss
%



\section{Motivation: Lösung einer linearen Differentialgleichung
\label{laplace:section:problemstellung}}
\rhead{Problemstellung}
Wir wollen eine Lösung $f(t)$ der linearen Differentialgleichung 
\begin{equation}
\frac{df}{dt} + \lambda f(t) = 0, \qquad ~~ \text{mit} \qquad f_{0} = f(t=0).
\label{laplace:lindgl}
\end{equation}
Wir wenden die Laplace-Transformation auf die Gleichung \eqref{laplace:lindgl} an. Mit der Laplace-Transformation folgt
\index{Laplace-Transformation}%
\begin{equation}
(sF(s) - f_{0}) + \lambda F(s) = 0.
\label{laplace:dgl}
\end{equation}
Die Gleichung \eqref{laplace:dgl} im Frequenzbereich kann nach $F(s)$ aufgelöst werden.
\index{Frequenzbereich}%
Es folgt somit
\begin{equation}
F(s) = \frac{f_{0}}{s + \lambda}.
\label{laplace:F(s)}
\end{equation}
Um die Laplace-Inverse von $F(s)$ zu finden, existieren für gewisse Funktionstypen tabellierte zugehörige Rücktransformationen.
\index{Laplace-Inverse}%
\index{Rücktransformation}%
Für die Funktion $F(s)$ in Gleichung \eqref{laplace:F(s)} ergibt sich zum Beispiel
\begin{equation}
\mathcal{L}^{-1}\{F(s)\}=\mathcal{L}^{-1}\biggl\{\frac{f_{0}}{s+\lambda}\biggr\} = f_{0}e^{-\lambda t}.
\end{equation}
Solche Tabellen sind jeweils in der Literatur zu finden, welche sich mit der Laplace-Transformation beschäftigt.
Wenn für die gegebene Funktion $F(s)$ keine zugehörige Funktion tabelliert ist, muss das Integral \eqref{laplace:riemanninversionsformel} wie in der Einleitung beschrieben numerisch berechnet werden.

