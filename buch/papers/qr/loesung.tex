%
% loesung.tex -- Beispiel-File für die Beschreibung der Loesung
%
% (c) 2020 Prof Dr Andreas Müller, Hochschule Rapperswil
%
\section{Lösung mit Givens-Rotationen
\label{qr:section:loesung}}
\index{Givens-Rotation}%
\rhead{Lösung}
Eine Drehung in der von $x_i$ und $x_j$ aufgespannten Ebene um den Winkel $\alpha$ kann mit der Matrix
\begin{equation}
D_{\alpha,i,j}=
\begin{pmatrix}
1     &\cdots&0     &\cdots&0     &\cdots&0\\
\vdots&\ddots&\vdots&      &\vdots&      &\vdots\\
0     &\cdots&c     &\cdots&-s    &\cdots&0\\
\vdots&      &\vdots&\ddots&\vdots&      &\vdots\\
0     &\cdots&s     &\cdots&c     &\cdots&0\\
\vdots&      &\vdots&      &\vdots&\ddots&\vdots\\
0     &\cdots&0     &\cdots&0     &\cdots&1
\end{pmatrix}
\end{equation}
beschrieben werden, wobei $c=\cos(\alpha)$ und $s=\sin(\alpha)$.
Drehungen dieser Art heissen Givens-Rotationen.

Es ist nun möglich, Elemente unterhalb der Diagonalen einer $l\times n $-Matrix $A$ mit geeigneten Givens-Rotationen auf Null zu bringen, um so eine obere Dreiecksmatrix $R$ zu erhalten.
Das Produkt aller inversen Drehmatrizen ist dann eine orthogonale Matrix $Q$, da alle Drehmatrizen $D_{\alpha_k,i,j}$ orthogonal sind.

Das wird nun an dieser Stelle im Detail betrachtet.
Zuerst wird mit Givens-Rotationen ($D_{\alpha_k, i, j}$ ist von der Grösse $l\times l$) die Matrix $R$ berechnet: 
\begin{equation*}
R = D_{\alpha_k,l,n-1}...D_{\alpha_2,3,1}D_{\alpha_1,2,1}A.		
\end{equation*}


Die erste Drehmatrix soll bewirken, dass das Element in der zweiten Zeile und ersten Spalte nach der Multiplikation 
\begin{equation*}
	D_{\alpha_1, 2,1}A=
	\begin{pmatrix}
		c_1   &-s_1  &\cdots &0\\
		\color{red}s_1   & \color{red}c_1  &\color{red}\cdots &\color{red}0\\
		\vdots&\vdots&\ddots&\vdots\\
		0     &0     &\cdots&1\\
	\end{pmatrix}
	\begin{pmatrix}
		\color{red}a_{11}&a_{12}&\cdots&a_{1n}\\
		\color{red}a_{21}&a_{22}&\cdots&a_{2n}\\
		\color{red}\vdots&\vdots&\ddots&\vdots\\
		\color{red}a_{l1}&a_{l2}&\cdots&a_{ln}
	\end{pmatrix}
\end{equation*}
verschwindet.
Interessant sind dabei nur die rot eingefärbte Zeile und Spalte, deren Multiplikation Null ergeben muss:
\begin{equation*}
	s_1 a_{11}+c_1 a_{21}=0,
\end{equation*}
und da $s_1=\sin(\alpha_1)$ und $c_1=\cos(\alpha_1)$, kennt man noch die Nebenbedingung
\begin{equation*}
	s_1^2+c_1^2=1.
\end{equation*} 
Dieses quadratische Gleichungssystem mit zwei Unbekannten und zwei Gleichungen kann somit nach $s_1$ und $c_2$ aufgelöst werden, wobei es wegen der quadratischen Nebenbedingung immer zwei Lösungen gibt.
Welche genommen wird spielt aber keine Rolle, da nur die Auswirkung der Drehung und nicht der eigentliche Drehwinkel relevant ist.

Die Matrix $D_{\alpha_1, 2,1}$ ist so vollständig bestimmt, ohne dass der Winkel $\alpha_1$ explizit bestimmt werden musste.
Dies kann so für alle weiteren Drehmatrizen $D_{\alpha_2, 3,1}$ bis $D_{\alpha_k, l,n-1}$ durchgeführt werden.

Auf die Matrix $Q$ kommt man dann mit
\begin{align*}
Q&=AR^{-1}\\&=A(D_{\alpha_k,l,n-1}...D_{\alpha_2,3,1}D_{\alpha_1,2,1}A)^{-1}\\&=
\underbrace{AA^{-1}}_{\displaystyle=E}D_{\alpha_1,2,1}^{-1}D_{\alpha_2,3,1}^{-1}... D_{\alpha_k, l,n-1}^{-1}=
D_{\alpha_1,2,1}^{T}D_{\alpha_2,3,1}^{T}...D_{\alpha_k, l,n-1}^{T},
\end{align*}
wobei im letzten Schritt ausgenutzt wurde, dass Drehmatrizen orthogonal sind.

Die Matrizen $Q$ und $R$ sind beide von der Grösse $l\times l$.
Für den Spezialfall $l=n$ können sie so stehen gelassen werden.
Für alle Fälle aber wo $l>n$ müssen die Dimensionen noch angepasst werden, um den Forderungen der $QR$-Zerlegung zu genügen.
In der Matrix $R$ stehen ab der $n.$ Zeile nur noch Nullen.
Diese Zeilen können somit weggelassen werden und folglich alle Einträge ab der $n$-ten Spalte von $Q$.
Als Matrizen ausgeschrieben:
\begin{align*}
A=QR
&=
\begin{pmatrix}
q_{11}&\cdots&q_{1n}&q_{1(n+1)}&\cdots&q_1l\\
\vdots&\ddots&\vdots&\vdots    &\ddots&\vdots\\
q_{l1}&\cdots&q_{ln}&q_{l(n+1)}&\cdots&q_{ll}
\end{pmatrix}
\begin{pmatrix}
r_{11}&r_{12}&\cdots&r_{1n}\\
0     &r_{22}&\cdots&r_{2n}\\
\vdots&\vdots&\ddots&\vdots\\
0     &0     &\cdots&r_{nn}\\
0     &0     &\cdots&0\\
\vdots&\vdots&\ddots&\vdots\\
0     &0     &\cdots&0
\end{pmatrix}\\
&=
\begin{pmatrix}
q_{11}&\cdots&q_{1n}&\\
\vdots&\ddots&\vdots&\\
q_{l1}&\cdots&q_{ln}\\
\end{pmatrix}
\begin{pmatrix}
r_{11}&r_{12}&\cdots&r_{1n}\\
0     &r_{22}&\cdots&r_{2n}\\
\vdots&\vdots&\ddots&\vdots\\
0     &0     &\cdots&r_{nn}
\end{pmatrix}.
\end{align*}
$Q$ ist daraufhin also eine $l\times n$-Matrize und $R$ ist von der Grösse $n\times n$, wie verlangt.

