%
% einleitung.tex -- Beispiel-File für die Einleitung
%
% (c) 2020 Prof Dr Andreas Müller, Hochschule Rapperswil
%
\section{Einleitung\label{quadratur:section:einleitung}}
\rhead{Einleitung}

Im Kapitel~\ref{chapter:integration}, Integration, wurden die 
Trapezregel und die Mittelpunktsregel für die numerische Integration, 
\index{numerische Integration}%
\index{Integration, numerische}%
auch Quadratur genannt, erklärt. 
\index{Quadratur}%
In diesem Kapitel wird eine weitere Methode, die Gauss-Quadratur, erarbeitet.
Die Gauss-Quadratur ist ein Verfahren, welches ein bestimmtes Integral der Form
\index{Gauss-Quadratur}%
\begin{equation}
    \int_{a}^{b} f(x) \,dx
\end{equation}
mit der approximierten Summenformel
\index{Summenformel}
\begin{equation} \label{quadratur:equation:quadraturapproxsumme}
    I = \sum_{i=0}^{n} A_i f(x_i)
\end{equation}
annähert, wobei die Stützstellen $x_i$ und die Gewichtung $A_i$ von der gewählten 
Methode abhängen. 
Es werden dabei im Vergleich mit ähnlichen Verfahren viel weniger Funktionsauswertungen benötigt.

Methoden für die Quadratur lassen sich in zwei Gruppen unterteilen: 
Newton-Cotes-Formeln und Gauss-Quadratur.
\index{Newton-Cotes-Formel}
Die im Kapitel \ref{buch:subsection:mittelpunkt} und \ref{buch:subsection:trapez} beschriebenen
Trapezregel und Mittelpunktsregel gehören zu der ersten Kategorie.
\index{Trapezregel}%
\index{Mittelpunktsregel}%
Newton-Cotes-Formeln lassen sich dadurch erkennen, dass die Stützstellen auf der $x$-Achse 
gleichmässig verteilt sind und eignen sich besonders für die Integration, wenn sich $f(x)$ 
effizient in kleinen Intervallen berechnen lässt oder bereits von Computern berechnet worden ist.







